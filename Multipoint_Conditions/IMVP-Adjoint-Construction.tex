\documentclass[11pt,reqno,oneside,a4paper]{article}
\usepackage{fancyhdr}
\usepackage{amsmath}
\usepackage{esint}
\usepackage{amsfonts}
\usepackage{amsthm}
\usepackage{amssymb}
\usepackage{amsbsy}
\usepackage{verbatim}
\usepackage{eucal}
\usepackage{mathrsfs}
\usepackage[hmargin = 1in,vmargin=1in]{geometry}
\usepackage[parfill]{parskip}
\usepackage{cite}

\setlength{\parskip}{2ex}
\pagestyle{fancy}

\lhead{Sultan}
\rhead{June 3, 2019}

\theoremstyle{plain} % default
\newtheorem{theorem}{Theorem}
\newtheorem{proposition}{Proposition}
\theoremstyle{definition}
\newtheorem{definition}[theorem]{Definition}
\theoremstyle{remark}
\newtheorem{remark}[theorem]{Remark}

\begin{document}
	
\subsection*{Construction of Adjoint Problem}
Consider a closed interval $[a,b].$ Fix $n\in\mathbb{N},$ and let the differential operator be defined as
\[ 
L  := \sum^n_{k=0} a_k(t)\left( \frac{d}{dt}\right)^k, \mbox{ where } a_k(t) \in C^{\infty}[a,b] \mbox{ and } a_n(t) \neq 0~ \forall t \in [a,b].
\]
Fix $k \in \mathbb{N},$ and let $\pi = \{ a = x_0 < x_1 < \ldots < x_k = b\}$ be a partition of $[a,b].$ Consider a homogeneous multipoint BVP of rank $m$
\[ \pi_m: Lq = 0, \qquad Uq = \vec{0},\]
where $U = (U_1, \ldots, U_m)$ is a multipoint boundary form with 
\[ 
U_i(q) = \sum^{k}_{l=1} \sum^{n-1}_{j=0}[\alpha_{ijl} f_l^{(j)}(x_{l-1}) + \beta_{ijl} f_l^{(j)}(x_{l})], \qquad i \in \{ 1, \ldots, m \}, 
\]
where $\alpha_{ijl}, \beta_{ijl} \in \mathbb{R}, q \in H^n(\pi),$ as given in Locker's paper \cite{locker}.
Our goal is to construct the adjoint multipoint BVP to $\pi_m$
\[ 
\pi^+_{2nk-m}: L^+ q = 0, \qquad U^+q = \vec{0},
\] 
where $L^+$ is the adjoint of $L,$ and $U^+$ is an appropriate multipoint boundary form.
\subsubsection*{Green's Formula}
For any $f,g \in H^n(\pi),$ application of Green's formula yields
\[ \langle Lf,g\rangle - \langle f,L^+ g\rangle = \sum_{l=1}^{k}\sum_{p,q=0}^{n-1}[F_{p\,q}(x_l) f_l^{(p)}(x_l)g_l^{(q)}(x_l) - F_{p\,q}(x_{l-1})f_l^{(p)}(x_{l-1})g_l^{(q)}(x_{l-1})], \]
where $F(t)$ denotes an $n\times n$ boundary matrix at the point $t \in [a,b].$ From \cite[p. 1286]{dunford}, the entries of $F(t)$ are given by
\begin{equation*}
\begin{aligned}
F_{p\,q}(t) &= \sum^{n-p-1}_{k = j} (-1)^k \binom{k}{j} \left( \frac{d}{dt}\right)^{k-j} a_{p+k+1}(t), &&\qquad p + q< n - 1\\
F_{p\,q}(t) &= (-1)^q  a_{n}(t), &&\qquad p + q= n - 1\\
F_{p\,q}(t) &= 0, &&\qquad p + q > n - 1.
\end{aligned}
\end{equation*}
Observe that since $\mathrm{det}F(t) = (a_0(t))^n \neq 0,$ the matrix $F(t)$ is non-singular. 

We let 
\[ 
[fg]_l(t) = \sum_{p,q=0}^{n-1}F_{p\,q}(t)f_l^{(p)}(t)g_l^{(q)}(t),
\]
so that the Green's formula can be rewritten as 
\[ \langle Lf,g\rangle - \langle f,L^+ g\rangle = \sum_{l=1}^{k} [fg]_l(x_l) - [fg]_l(x_{l-1}).\]
Now, we seek another matrix $\widehat{F}_l,$ with which we can associate a \emph{semibilinear} form $\mathcal{S}_l$. We derive this matrix in the same way as in Linda's capstone \cite{linfan}. First, observe that 
\begin{align*}
[fg]_l(t) = \sum_{p,q=0}^{n-1}F_{p\,q}(t) f_l^{(p)}(t)g_l^{(q)}(t) &= \sum_{p,q=0}^{n-1}\left[F_{p\,q} f_l^{(p)} g_l^{(q)}\right](t)  \\
&= \sum_{q=0}^{n-1}\left[ \left(\sum_{p=0}^{n-1} F_{p\,q} f_l^{(p)}\right) g_l^{(q)}\right](t) \\
&= F(t) \vec{f_l}(t) \cdot \vec{g_l}(t), 
\end{align*}
where $\vec{f_l} = (f_l, \ldots, f_l^{(n-1)})$ and $\vec{g_l} = (g_l, \ldots, g_l^{(n-1)}).$ We use this to obtain:
\begin{align*}
[fg]_l(x_l) - [fg]_l(x_{l-1}) &= F(x_l) \vec{f_l}(x_l) \cdot \vec{g_l}(x_l) - F(x_{l-1}) \vec{f_l}(x_{l-1}) \cdot \vec{g_l}(x_{l-1}) \\
&= \begin{bmatrix}
- F(x_{l-1}) & 0_{n\times n} \\
0_{n\times n} &  F(x_{l}) \\
\end{bmatrix}
\begin{bmatrix}
f_l(x_{l-1})  \\
\vdots \\
f_l^{(n-1)}(x_{l-1}) \\
f_l(x_{l})  \\
\vdots \\
f_l^{(n-1)}(x_{l}) 
\end{bmatrix}
\cdot
\begin{bmatrix}
g_l(x_{l-1})  \\
\vdots \\
g_l^{(n-1)}(x_{l-1}) \\
g_l(x_{l})  \\
\vdots \\
g_l^{(n-1)}(x_{l}) 
\end{bmatrix} \\
&=: \widehat{F}_l(x_{l-1}, x_l)  
\begin{bmatrix}
\vec{f_l}(x_{l-1})  \\
\vec{f_l}(x_{l}) 
\end{bmatrix}
\cdot
\begin{bmatrix}
\vec{g_l}(x_{l-1})  \\
\vec{g_l}(x_{l}) 
\end{bmatrix} \\
\end{align*}
Note that since $F(x_l)$ is nonsingular for all $x_l,$ it follows that $\widehat{F}_l(x_{l-1}, x_l)$ is also non-singular for all $x_l$. Finally, with the above in mind, we obtain
\begin{align*}
\langle Lf,g\rangle - \langle f,L^+ g\rangle = \sum_{l=1}^{k} [fg]_l(x_l) - [fg]_l(x_{l-1}) &=  \sum_{l=1}^{k} \widehat{F}_l(x_{l-1}, x_l)  
\begin{bmatrix}
\vec{f_l}(x_{l-1})  \\
\vec{f_l}(x_{l})  \\
\end{bmatrix}
\cdot
\begin{bmatrix}
\vec{g_l}(x_{l-1})  \\
\vec{g_l}(x_{l})  \\
\end{bmatrix} .
\end{align*}
Now, expansion of the sum yields
\begin{equation*}
\begin{aligned}
&\sum_{l=1}^{k} \widehat{F}_l(x_{l-1}, x_l) 
\begin{bmatrix}
\vec{f_l}(x_{l-1})  \\
\vec{f_l}(x_{l})  \\
\end{bmatrix}
\cdot
\begin{bmatrix}
\vec{g_l}(x_{l-1})  \\
\vec{g_l}(x_{l})  \\
\end{bmatrix} \\
&= 
\widehat{F}_1(x_0, x_1) \begin{bmatrix}
\vec{f_1}(x_0)  \\
\vec{f_1}(x_1)  \\
\end{bmatrix}
\cdot
\begin{bmatrix}
\vec{g_1}(x_0)  \\
\vec{g_1}(x_1)  \\
\end{bmatrix}
+ \ldots + 
\widehat{F}_k(x_{k-1}, x_k) \begin{bmatrix}
\vec{f_k}(x_{k-1})  \\
\vec{f_k}(x_k)  \\
\end{bmatrix}
\cdot
\begin{bmatrix}
\vec{g_k}(x_{k-1})  \\
\vec{g_k}(x_k)  \\
\end{bmatrix} \\
&= \underbrace{
\begin{bmatrix}
\widehat{F}_1(x_0, x_1) & 0 & \ldots & 0 & 0 \\
0 & \widehat{F}_2(x_1, x_2) & \ldots & 0 & 0 \\
\vdots & \vdots & \ddots & \vdots & \vdots \\
0 & 0 & \ldots & \widehat{F}_{k-1}(x_{k-2}, x_{k-1}) & 0 \\
0 & 0 & \ldots & 0 & \widehat{F}_k(x_{k-1}, x_k)
\end{bmatrix}}_\text{2nk$\times$2nk}
\begin{bmatrix}
\vec{f_1}(x_{0})  \\
\vec{f_1}(x_1) \\
\vec{f_2}(x_1)  \\
\vec{f_2}(x_2) \\
\vdots \\
\vec{f_k}(x_{k-1})  \\
\vec{f_k}(x_k)
\end{bmatrix}
\cdot 
\begin{bmatrix}
\vec{g_1}(x_{0})  \\
\vec{g_1}(x_1) \\
\vec{g_2}(x_1)  \\
\vec{g_2}(x_2) \\
\vdots \\
\vec{g_k}(x_{k-1})  \\
\vec{g_k}(x_k)
\end{bmatrix} \\
&=: \mathcal{S} \left( \begin{bmatrix}
\vec{f_1}(x_{0})  \\
\vec{f_1}(x_1) \\
\vdots \\
\vec{f_k}(x_{k-1})  \\
\vec{f_k}(x_k)
\end{bmatrix},  
\begin{bmatrix}
\vec{g_1}(x_{0})  \\
\vec{g_1}(x_1) \\
\vdots \\
\vec{g_k}(x_{k-1})  \\
\vec{g_k}(x_k)
\end{bmatrix}
\right),
\end{aligned}
\end{equation*}
i.e. we managed to express the Green's Formula as in terms of a semibilinear form $\mathcal{S}.$ Observe that the matrix on the second-last line is a block matrix where block matrices are 2n$\times$2n.
\subsubsection*{Boundary-Form Formula}
We turn to characterising an adjoint multipoint boundary condition using an extension of boundary form formula that Linda derived in her work.
First, recall that the multipoint boundary conditions are of the form
\[ 
Uq = \begin{bmatrix} U_1(q) \\ \vdots \\U_m(q) \end{bmatrix} = \vec{0},
\]
with 
\[ 
U_i(q) = \sum^{k}_{l=1} \sum^{n-1}_{j=0}[\alpha_{ijl} q_l^{(j)}(x_{l-1}) + \beta_{ijl} q_l^{(j)}(x_{l})], \qquad i \in \{ 1, \ldots, m \}, ~ \alpha_{ijl}, \beta_{ijl} \in \mathbb{R}.
\]
Note that $U_1, \ldots, U_m$ are linearly independent when $\sum^m_{i=1} c_i U_i q = 0$ if and only if $c_i = 0.$ When $U_1, \ldots, U_m$ are linearly independent, we say that $U$ has full rank $m.$ For now, suppose that $U$ has full rank, and define 
\begin{equation*}
\vec{q_l} = 
\begin{bmatrix}
q_l \\
q'_l \\
\vdots \\
 q_l^{(n-1)} 
\end{bmatrix}, M_l = 
\begin{bmatrix}
\alpha_{1~0~l} & \alpha_{1~1~l} & \ldots & \alpha_{1~(n-1)~l} \\
\alpha_{2~0~l} & \alpha_{2~1~l} & \ldots & \alpha_{2~(n-1)~l} \\
\vdots & \vdots & \ddots & \vdots \\
\alpha_{m~0~l} & \alpha_{m~1~l} & \ldots & \alpha_{m~(n-1)~l} \\
\end{bmatrix}, 
N_l = 
\begin{bmatrix}
\beta_{1~0~l} & \beta_{1~1~l} & \ldots & \beta_{1~(n-1)~l} \\
\beta_{2~0~l} & \beta_{2~1~l} & \ldots & \beta_{2~(n-1)~l} \\
\vdots & \vdots & \ddots & \vdots \\
\beta_{m~0~l} & \beta_{m~1~l} & \ldots & \beta_{m~(n-1)~l} \\
\end{bmatrix}
\end{equation*}
Then, 
\begin{align*}
Uq &= \begin{bmatrix} U_1(q) \\ \vdots \\U_m(q) \end{bmatrix} \\&= 
\sum^k_{l=1} \sum^{n-1}_{j=0} \begin{bmatrix} \alpha_{1~j~l}  \\ \vdots \\\alpha_{m~j~l}\end{bmatrix}  q_l^{(j)}(x_{l-1}) + \begin{bmatrix} \beta_{1~j~l} \\ \vdots \\\beta_{m~j~l}\end{bmatrix} q_l^{(j)}(x_{l}) \\
&= 
\sum^k_{l=1} 
\begin{bmatrix}
\alpha_{1~0~l}  & \ldots & \alpha_{1~(n-1)~l} \\
\vdots & \ddots & \vdots \\
\alpha_{m~0~l} &  \ldots & \alpha_{m~(n-1)~l} \\
\end{bmatrix}
\begin{bmatrix}
q_l(x_{l-1})  \\
\vdots \\
 q_l^{(n-1)}(x_{l-1})  
\end{bmatrix} + 
\begin{bmatrix}
\beta_{1~0~l} & \ldots & \beta_{1~(n-1)~l} \\
\vdots & \ddots & \vdots \\
\beta_{m~0~l} & \ldots & \beta_{m~(n-1)~l} \\
\end{bmatrix}
\begin{bmatrix}
q_l(x_l)  \\
\vdots \\
 q_l^{(n-1)}(x_l)  
\end{bmatrix} \\
&= \sum^k_{l=1} M_l \vec{q_l}(x_{l-1}) + N_l \vec{q_l}(x_l),
\end{align*}
where $M_l, N_l$ are $m \times n$ matrices. 
In addition, letting 
\[ 
(M_l : N_l) = 
\begin{bmatrix}
\alpha_{1~0~l} & \alpha_{1~1~l} & \ldots & \alpha_{1~(n-1)~l} & \beta_{1~0~l} & \beta_{1~1~l} & \ldots & \beta_{1~(n-1)~l} \\
%\alpha_{2~0~l} & \alpha_{2~1~l} & \ldots & \alpha_{2~(n-1)~l} & \beta_{2~0~l} & \beta_{2~1~l} & \ldots & \beta_{2~(n-1)~l} \\
\vdots & \vdots & \ddots & \vdots & \vdots & \vdots & \ddots & \vdots \\
\alpha_{m~0~l} & \alpha_{m~1~l} & \ldots & \alpha_{m~(n-1)~l} & \beta_{m~0~l} & \beta_{m~1~l} & \ldots & \beta_{m~(n-1)~l}
\end{bmatrix},
\]
we can write 
\[ Uq = \sum^k_{l=1}(M_l:N_l) 
\begin{bmatrix}
\vec{q_l}(x_{l-1})  \\
\vec{q_l}(x_{l})
\end{bmatrix}.
\]
Thus we have found 2 compact ways of writing the multipoint boundary forms. Next, we extend the notion of a complementary boundary form.
\newline
\begin{definition}
If $U = (U_1, \ldots, U_m)$ is any multipoint boundary form with $\mathrm{rank}(U) = m,$ and $U_c = (U_{m+1}, \ldots, U_{2nk})$ is a multipoint boundary form with $\mathrm{rank}(U_c) = 2nk-m$ such that $\mathrm{rank}(U_{1}, \ldots, U_{2nk}) = 2nk,$ then $U$ and $U_c$ are \textbf{complementary multipoint boundary forms}. 
\end{definition}
Note that extending $U_1, \ldots, U_m$ to $U_{1}, \ldots, U_{2nk}$ is equivalent to embedding the matrices $M_l, N_l$ in a $2nk \times 2nk$ non-singular matrix, i.e. we can write
\begin{align*}
\begin{bmatrix}
Uq \\
U_c q
\end{bmatrix} 
&=
\sum^k_{l=1}
\begin{bmatrix}
M_l & N_l \\
\overline{M}_l & \overline{N}_l 
\end{bmatrix} 
\begin{bmatrix}
\vec{q_l}(x_{l-1})  \\
\vec{q_l}(x_{l})
\end{bmatrix} \\
&= 
\begin{bmatrix}
M_1 & N_1 \\
\overline{M}_1 & \overline{N}_1
\end{bmatrix} 
\begin{bmatrix}
\vec{q_1}(x_{0})  \\
\vec{q_1}(x_1)
\end{bmatrix} 
+ 
\begin{bmatrix}
M_2 & N_2 \\
\overline{M}_2 & \overline{N}_2
\end{bmatrix} 
\begin{bmatrix}
\vec{q_2}(x_1)  \\
\vec{q_2}(x_2)
\end{bmatrix} 
+ \ldots +
\begin{bmatrix}
M_k & N_k \\
\overline{M}_k & \overline{N}_k
\end{bmatrix} 
\begin{bmatrix}
\vec{q_k}(x_{k-1})  \\
\vec{q_k}(x_k)
\end{bmatrix} \\
&= 
\underbrace{
\begin{bmatrix}
M_1 & N_1 & M_2 & N_2 & \ldots & M_k & N_k \\
\overline{M}_1 & \overline{N}_1 & \overline{M}_2 & \overline{N}_2 & \ldots & \overline{M}_k & \overline{N}_k
\end{bmatrix} }_\text{$2nk \times 2nk$}
\underbrace{\begin{bmatrix}
\vec{q_1}(x_{0})  \\
\vec{q_1}(x_1) \\
\vec{q_2}(x_1)  \\
\vec{q_2}(x_2) \\
\vdots \\
\vec{q_k}(x_{k-1})  \\
\vec{q_k}(x_k)
\end{bmatrix}}_\text{$2nk\times 1$} \\
&=:
H
\begin{bmatrix}
\vec{q_1}(x_{0})  \\
\vec{q_1}(x_1) \\
\vec{q_2}(x_1)  \\
\vec{q_2}(x_2) \\
\vdots \\
\vec{q_k}(x_{k-1})  \\
\vec{q_k}(x_k)
\end{bmatrix}.
\end{align*}
where $\mathrm{rank}(H) = 2nk$ and $\overline{M}_l, \overline{N}_l$ are $2nk-m \times n$ matrices. Just like the boundary form formula proven by Linda, the multipoint boundary form formula is motivated by the desire to express Green's formula as a combination of boundary forms $U$ and $U_c.$ Namely, we have:
\newline
\begin{theorem}[Multipoint Boundary Form Formula]\label{mbff}
Given any boundary form $U$ of rank $m,$ and any complementary form $U_c,$ there exist unique boundary forms $U^+_c, U^+$ of rank $m$ and $2n-m,$ respectively, such that
\[ 
\sum_{l=1}^{k} [fg]_l(x_l) - [fg]_l(x_{l-1}) = Uf\cdot U^+_c g + U_c f \cdot U^+ g.
\]
\end{theorem}
We will use the following proposition from Linda's capstone \cite{linfan} in the proof of Theorem \ref{mbff}:
\newline
\begin{proposition}[Prop. 2.12 in Linda's capstone]\label{linfan-2.12}
Let $\mathcal{S}$ be the semibilinear form associated with a nonsingular matrix $S.$ Suppose $\vec{f} := Ff$ where $F$ is a nonsingular matrix. Then, there exists a unique nonsingular matrix $G$ such that if $\vec{g} =Gg,$ then $\mathcal{S}(f,g) = \vec{f}\cdot \vec{g}$ for all $f,g.$
\end{proposition}
\begin{proof}
We prove Theorem \ref{mbff}. 
First, we have 
\[ 
\begin{bmatrix}
Uf \\
U_c f
\end{bmatrix} = 
H
\begin{bmatrix}
\vec{f_1}(x_{0})  \\
\vec{f_1}(x_1) \\
\vdots \\
\vec{f_k}(x_{k-1})  \\
\vec{f_k}(x_k)
\end{bmatrix}.
\]
As shown in the subsection on Green's formula, we can write 
\begin{align*}
\sum_{l=1}^{k} [fg]_l(x_l) - [fg]_l(x_{l-1}) &= \sum_{l=1}^{k} \widehat{F}_l(x_{l-1}, x_l) \begin{bmatrix}
\vec{f_l}(x_{l-1})  \\
\vec{f_l}(x_{l})  \\
\end{bmatrix}
\cdot
\begin{bmatrix}
\vec{g_l}(x_{l-1})  \\
\vec{g_l}(x_{l})  \\
\end{bmatrix}  \\
&= \mathcal{S} \left( \begin{bmatrix}
\vec{f_1}(x_{0})  \\
\vec{f_1}(x_1) \\
\vdots \\
\vec{f_k}(x_{k-1})  \\
\vec{f_k}(x_k)
\end{bmatrix},  
\begin{bmatrix}
\vec{g_1}(x_{0})  \\
\vec{g_1}(x_1) \\
\vdots \\
\vec{g_k}(x_{k-1})  \\
\vec{g_k}(x_k)
\end{bmatrix}
\right),
\end{align*}
Now, by Proposition \ref{linfan-2.12}, there exists a unique $2nk \times 2nk $ nonsingular matrix $J$ such that 
$$\mathcal{S} \left( \begin{bmatrix}
\vec{f_1}(x_{0})  \\
\vec{f_1}(x_1) \\
\vdots \\
\vec{f_k}(x_{k-1})  \\
\vec{f_k}(x_k)
\end{bmatrix},  
\begin{bmatrix}
\vec{g_1}(x_{0})  \\
\vec{g_1}(x_1) \\
\vdots \\
\vec{g_k}(x_{k-1})  \\
\vec{g_k}(x_k)
\end{bmatrix}
\right)
= H
\begin{bmatrix}
\vec{f_1}(x_{0})  \\
\vec{f_1}(x_1) \\
\vdots \\
\vec{f_k}(x_{k-1})  \\
\vec{f_k}(x_k)
\end{bmatrix} \cdot 
J\begin{bmatrix}
\vec{g_1}(x_{0})  \\
\vec{g_1}(x_1) \\
\vdots \\
\vec{g_k}(x_{k-1})  \\
\vec{g_k}(x_k)
\end{bmatrix}.$$ Let $U^+, U^+_c$ be such that 
\[
\begin{bmatrix}
U^+_cg \\
U^+ g
\end{bmatrix} = J\begin{bmatrix}
\vec{g_1}(x_{0})  \\
\vec{g_1}(x_1) \\
\vdots \\
\vec{g_k}(x_{k-1})  \\
\vec{g_k}(x_k)
\end{bmatrix}.
\]
Then, we obtain 
\begin{align*}
\sum_{l=1}^{k} [fg]_l(x_l) - [fg]_l(x_{l-1})  = \mathcal{S} \left( \begin{bmatrix}
\vec{f_1}(x_{0})  \\
\vec{f_1}(x_1) \\
\vdots \\
\vec{f_k}(x_{k-1})  \\
\vec{f_k}(x_k)
\end{bmatrix},  
\begin{bmatrix}
\vec{g_1}(x_{0})  \\
\vec{g_1}(x_1) \\
\vdots \\
\vec{g_k}(x_{k-1})  \\
\vec{g_k}(x_k)
\end{bmatrix}
\right)
&= H
\begin{bmatrix}
\vec{f_1}(x_{0})  \\
\vec{f_1}(x_1) \\
\vdots \\
\vec{f_k}(x_{k-1})  \\
\vec{f_k}(x_k)
\end{bmatrix} \cdot 
J\begin{bmatrix}
\vec{g_1}(x_{0})  \\
\vec{g_1}(x_1) \\
\vdots \\
\vec{g_k}(x_{k-1})  \\
\vec{g_k}(x_k)
\end{bmatrix} \\
&= \begin{bmatrix}
Uf \\
U_c f
\end{bmatrix} \cdot 
\begin{bmatrix}
U^+_cg \\
U^+ g
\end{bmatrix} \\
&=  Uf\cdot U^+_c g + U_c f \cdot U^+ g,
\end{align*}
which completes the proof.
\end{proof}
Theorem \ref{mbff} allows us to define an adjoint multipoint boundary form. Namely, 
\newline
\begin{definition}
Suppose $U = (U_1, \ldots, U_m)$ is a multipoint boundary form with $\mathrm{rank}(U) = m,$ along with the condition that $Uq = \vec{0}$ for functions $q \in H^n(\pi).$ If $U^+$ is any boundary form with  $\mathrm{rank}(U^+) = 2nk-m,$ determined as in Theorem \ref{mbff}, then the equation 
\[ 
U^+q = \vec{0}
\]
is an \textbf{adjoint multipoint boundary form} to $Uq = \vec{0}.$
\end{definition}

In turn, the above lets us define the adjoint multipoint problem:
\newline
\begin{definition}
Suppose $U = (U_1, \ldots, U_m)$ is a multipoint boundary form with $\mathrm{rank}(U) = m.$ Then, the problem of solving 
\[ \pi_m: Lq = 0, \qquad Uq = \vec{0},\] 
is called a homogeneous multipoint boundary value problem of rank $m.$ The problem of solving 
\[ \pi_{2nk-m}^+: L^+q = 0, \qquad U^+q = \vec{0},\] 
is an \textbf{adjoint multipoint boundary value problem} to $\pi_m.$
\end{definition}

The preceding construction allows us to state the following:
\newline
\begin{proposition}
Let $f,g \in C^n[a,b]$ with $Uf = \vec{0}$ and $U^+g = \vec{0}.$ Then, $\langle Lf, g\rangle = \langle f, L^+g\rangle.$
\end{proposition}
\begin{proof}
We apply Green's formula and multipoint boundary form formula:
\[ \langle Lf, g\rangle - \langle f, L^+g\rangle = \sum^k_{l=1}  [fg]_l(x_l) - [fg]_l(x_{l-1}) = Uf\cdot U^+_c g + U_c f \cdot U^+ g = \vec{0}\cdot U^+_c g + U_c f \cdot \vec{0} = 0. \qedhere\]
\end{proof}

\subsubsection*{Checking Adjointness}

Finally, we extend Theorem 2.19 on Linda's Capstone \cite{linfan}. 
\newline
\begin{theorem}
The boundary condition $U^+f = \vec{0}$ is adjoint to $Uf = \vec{0}$ if and only if \[ M_lF^{-1}(x_{l-1})P_l = N_l F^{-1}(x_l)Q_l \] for all $l = 1, \ldots, k,$ where $F(t)$ is the $n\times n$ matrix as given in Green's Formula subsection.  
\end{theorem}
Recall that just how $U$ is associated with a collection of $2k$ $m\times n$ matrices $M_l, N_l,$ so is $U^+$ associated with $2k$ $n\times(2n-m)$ matrices $P,Q$ such that $(P^{*}:Q^*)$ has rank $2n-m$ and 
\[ U^+q = \sum^k_{l=1} P^*_l \vec{q}_l(x_{l-1}) + Q^*_l \vec{q}_l(x_l). \]
\begin{proof}
Let $\vec{f_l} = (f_l, \ldots, f_l^{(n-1)})$ and $\vec{g_l} = (g_l, \ldots, g_l^{(n-1)}).$ Suppose that $U^+f = \vec{0}$ is adjoint to $Uf = \vec{0}.$ By definition of adjoint multipoint boundary condition, $U^+$ is determined as in Theorem \ref{mbff}. Thus, in determining $U^+,$ there exist multipoint boundary forms $U_c, U_c^+$ of rank $2nk-m$ and $m$ respectively, such that the multipoint boundary form formula holds. As such, let 
\begin{align*}
U_c f &= \sum^k_{l=1} \overline{M}_l \vec{f}_l(x_{l-1}) + \overline{N}_l  \vec{f}_l(x_l), \quad &&\mathrm{rank}(\overline{M}_l : \overline{N}_l) =  2nk - m \\
U_c^+ g &= \sum^k_{l=1} \overline{P}_l^* \vec{g}_l(x_{l-1}) + \overline{Q}_l^*  \vec{g}_l(x_l), \quad &&\mathrm{rank}( \overline{P}_l^* :  \overline{Q}_l^*) = m
\end{align*}
First, note that in the context of semibilinear form, we have $S f \cdot g = f \cdot S^* g,$ as given in Proposition 2.11 of Linda's capstone \cite[p.18]{linfan}. We use this to rewrite the multipoint boundary form formula as follows:
\begin{align*}
\sum^k_{l=1} [fg]_l(x_l) - [fg]_l(x_{l-1}) &= Uf\cdot U^+_c g + U_c f \cdot U^+ g \\
&= \left(\sum^k_{l=1} M_l \vec{f_l}(x_{l-1}) + N_l \vec{f_l}(x_l)\right)\cdot \left( \sum^m_{l=1} (\overline{P}_l)^* \vec{g}_l(x_{l-1}) + (\overline{Q}_l)^*  \vec{g}_l(x_l)  \right) \\
&+ \left(  \sum^k_{l=1} \overline{M}_l \vec{f}_l(x_{l-1}) + \overline{N}_l  \vec{f}_l(x_l) \right) \cdot \left( \sum^m_{l=1} P^*_l \vec{g}_l(x_{l-1}) + Q^*_l \vec{g}_l(x_l) \right) \\
&= \sum^k_{l=1} \Bigg(\left( M_l \vec{f_l}(x_{l-1}) + N_l \vec{f_l}(x_l)\right) \cdot \left( \overline{P}_l^* \vec{g}_l(x_{l-1}) + \overline{Q}_l^*  \vec{g}_l(x_l)  \right) \\
&+ \left( \overline{M}_l \vec{f}_l(x_{l-1}) + \overline{N}_l  \vec{f}_l(x_l) \right) \cdot \Big( P^*_l \vec{g}_l(x_{l-1}) + Q^*_l \vec{g}_l(x_l) \Big) \Bigg).
\end{align*}
Thus, expanding the $l$-th term, using that $S f \cdot g = f \cdot S^* g,$ and collecting similar terms, we can write the $l$-th term as
\begin{equation}\label{eq1}
\begin{aligned}
~[fg]_l(x_l) - [fg]_l(x_{l-1}) = (\overline{Q}_l N_l + Q_l \overline{N}_l)  \vec{f}_l(x_l)  \cdot  \vec{g}_l(x_l) &+  (\overline{P}_l N_l  + P_l\overline{N}_l ) \vec{f}_l(x_l)  \cdot  \vec{g}_l(x_{l-1}) \\
+   (\overline{Q}_l  M_l +Q_l \overline{M}_l) \vec{f_l}(x_{l-1}) \cdot\vec{g}_l(x_l) &+  (\overline{P}_l M_l+P_l \overline{M}_l) \vec{f_l}(x_{l-1}) \cdot  \vec{g}_l(x_{l-1}) 
\end{aligned}
\end{equation}
Recall from Green's formula subsection that the $l$-th term of the sum can be written
\begin{equation}\label{eq2}
[fg]_l(x_l) - [fg]_l(x_{l-1}) = F(x_l) \vec{f_l}(x_l) \cdot \vec{g_l}(x_l) - F(x_{l-1}) \vec{f_l}(x_{l-1}) \cdot \vec{g_l}(x_{l-1}) 
\end{equation}
Term-by-term comparison of \eqref{eq1} and \eqref{eq2} reveals that 
\begin{align*}
\overline{P}_l M_l+P_l \overline{M}_l &= - F(x_{l-1}), \qquad &&\overline{P}_l N_l  + P_l\overline{N}_l = 0_{n\times n}, \\
\overline{Q}_l  M_l +Q_l \overline{M}_l &= 0_{n\times n},  \qquad &&\overline{Q}_l N_l + Q_l \overline{N}_l = F(x_l).
\end{align*}
Since the boundary matrix $F$ is nonsingular on $[a,b],$ $F$ is invertible. Thus, we have 
\begin{align*}
- F(x_{l-1})^{-1} \overline{P}_l M_l - F(x_{l-1})^{-1} P_l \overline{M}_l &= E_n, \qquad -F(x_{l-1})^{-1}\overline{P}_l N_l  - F(x_{l-1})^{-1} P_l\overline{N}_l &&= 0_{n\times n}, \\
F(x_l)^{-1}\overline{Q}_l  M_l +F(x_l)^{-1}Q_l \overline{M}_l &= 0_{n\times n},  \qquad F(x_l)^{-1}\overline{Q}_l N_l + F(x_l)^{-1} Q_l \overline{N}_l &&= E_n.
\end{align*}
Using the systems notation, we have 
\[ 
\begin{bmatrix}
- F(x_{l-1})^{-1} \overline{P}_l & - F(x_{l-1})^{-1} P_l \\
F(x_l)^{-1}\overline{Q}_l & F(x_l)^{-1}Q_l 
\end{bmatrix}
\begin{bmatrix}
M_l & N_l \\
\overline{M}_l & \overline{N}_l
\end{bmatrix} = 
\begin{bmatrix}
E_n & 0_{n\times n} \\
0_{n\times n} & E_n.
\end{bmatrix}.
\]
Since $\begin{bmatrix}
M_l & N_l \\
\overline{M}_l & \overline{N}_l
\end{bmatrix}$ has full rank, two matrices on the left must be inverses of each other. Thus, we can write
\[ 
\begin{bmatrix}
M_l & N_l \\
\overline{M}_l & \overline{N}_l
\end{bmatrix} 
\begin{bmatrix}
- F(x_{l-1})^{-1} \overline{P}_l & - F(x_{l-1})^{-1} P_l \\
F(x_l)^{-1}\overline{Q}_l & F(x_l)^{-1}Q_l
\end{bmatrix}
= 
\begin{bmatrix}
E_{m\times m} & 0_{m\times 2nk-m} \\
0_{2nk-m\times m} & E_{2nk-m\times 2nk-m} .
\end{bmatrix},
\] 
which means that 
\[ 
-  M_l  F(x_{l-1})^{-1} P_l + N_l F(x_l)^{-1}Q_l   =  0_{m\times 2nk-m} \implies M_l  F(x_{l-1})^{-1} P_l = N_l F(x_l)^{-1}Q_l.
\]
Since $l$ is arbitrary, we conclude that the above holds for all $l = 1, \ldots, k.$

Now, let $U_1^+$ be a multipoint boundary form of rank $2nk-m$ such that 
\[ 
U_1^+ g = \sum^k_{l=1} (P_1)^*_l \vec{g}_l(x_{l-1}) + (Q_1)^*_l \vec{g}_l(x_l),
\]
for an appropriate collection of matrices $(P_1)^*_l, (Q_1)^*_l,$ with 
\[\mathrm{rank}((P_1)^*_l : (Q_1)^*_l) = 2nk-m, \qquad l = 1,\ldots, k.\]
 Suppose that 
\[ M_l  F(x_{l-1})^{-1} (P_1)_l = N_l F(x_l)^{-1}(Q_1)_l, \qquad l = 1,\ldots, k \]
holds. Now, fix $l$ and let $u_l$ be a $2n\times 1$ vector. Then, there exist $2nk-m$ linearly independent solutions of the system $(M_l : N_l)_{m\times 2n}u_l = \vec{0}.$ By assumption, we have 
\[
-  M_l  F(x_{l-1})^{-1} (P_1)_l + N_l F(x_l)^{-1}(Q_1)_l   =  0_{m\times (2nk-m)},
\]
so that 
\begin{equation}\label{system}
(M_l : N_l)_{m\times 2n} 
\begin{bmatrix}
-F(x_{l-1})^{-1} (P_1)_l \\
F(x_l)^{-1}(Q_1)_l
\end{bmatrix}_{2n\times(2nk-m)} = 0_{m \times(2nk-m)}. 
\end{equation}
This means that the $2nk-m$ columns of the matrix 
\[ (H_1)_l := \begin{bmatrix}
-F(x_{l-1})^{-1} (P_1)_l \\
F(x_l)^{-1}(Q_1)_l
\end{bmatrix}
\]
form the solution space of the system \eqref{system}. Since $\mathrm{rank}((P_1)^*_l : (Q_1)^*_l) = 2nk-m,$
\[ 
\mathrm{rank}
\begin{bmatrix}
(P_1)_l \\
(Q_1)_l
\end{bmatrix}
= 2nk-m.
\] 
Since $F(x_{l-1}), F(x_{l})$ are non-singular, $\mathrm{rank}(H_1) = 2nk - m.$
Now, if $U^+g = \sum^k_{l=1} P^*_l \vec{g}_l(x_{l-1}) + Q^*_l \vec{g}_l(x_l) = \vec{0}$ is a multipoint boundary condition adjoint to $Uf = \vec{0},$ then by multipoint boundary form formula we have that 
\begin{align*}
\begin{bmatrix}
Uf \\
U_c f
\end{bmatrix} \cdot 
\begin{bmatrix}
U^+_cg \\
U^+ g
\end{bmatrix} 
&= 
\sum^k_{l=1} 
\begin{bmatrix}
M_l  \vec{f}_l(x_{l-1}) + N_l \vec{f}_l(x_l) \\
\overline{M}_l \vec{f}_l(x_{l-1}) +\overline{N}_l \vec{f}_l(x_l) 
\end{bmatrix} \cdot 
\begin{bmatrix}
\overline{P}^*_l \vec{g}_l(x_{l-1}) + \overline{Q}^*_l \vec{g}_l(x_l) \\
P^*_l \vec{g}_l(x_{l-1}) + Q^*_l \vec{g}_l(x_l) 
\end{bmatrix} \\
&= 
\sum^k_{l=1} 
\begin{bmatrix}
M_l & N_l  \\
\overline{M}_l & \overline{N}_l 
\end{bmatrix} 
\begin{bmatrix}
\vec{f}_l(x_{l-1}) \\
\vec{f}_l(x_l) 
\end{bmatrix}
\cdot 
\begin{bmatrix}
\overline{P}_l & \overline{Q}_l \\
P_l & Q_l 
\end{bmatrix}^*
\begin{bmatrix}
\vec{g}_l(x_{l-1}) \\
\vec{g}_l(x_l) 
\end{bmatrix}\\
&=
\sum^k_{l=1} 
\begin{bmatrix}
\overline{P}_l & P_l \\
\overline{Q}_l & Q_l 
\end{bmatrix} 
\begin{bmatrix}
M_l & N_l  \\
\overline{M}_l & \overline{N}_l 
\end{bmatrix} 
\begin{bmatrix}
\vec{f}_l(x_{l-1}) \\
\vec{f}_l(x_l) 
\end{bmatrix}
\cdot 
\begin{bmatrix}
\vec{g}_l(x_{l-1}) \\
\vec{g}_l(x_l) 
\end{bmatrix}.
\end{align*}
In addition, recall from Green's formula subsection that
\[ 
\begin{bmatrix}
Uf \\
U_c f
\end{bmatrix} \cdot 
\begin{bmatrix}
U^+_cg \\
U^+ g
\end{bmatrix} 
=
\sum^k_{l=1} 
 \begin{bmatrix}
- F(x_{l-1}) & 0_{n\times n} \\
0_{n\times n} &  F(x_{l}) \\
\end{bmatrix}
\begin{bmatrix}
\vec{f_l}(x_{l-1})  \\
\vec{f_l}(x_{l})  \\
\end{bmatrix}
\cdot
\begin{bmatrix}
\vec{g_l}(x_{l-1})  \\
\vec{g_l}(x_{l})  \\
\end{bmatrix}.\]
Combining the above yields
\[ 
\begin{bmatrix}
- F(x_{l-1}) & 0_{n\times n} \\
0_{n\times n} &  F(x_{l}) \\
\end{bmatrix} = 
\begin{bmatrix}
\overline{P}_l & P_l \\
\overline{Q}_l & Q_l 
\end{bmatrix} 
\begin{bmatrix}
M_l & N_l  \\
\overline{M}_l & \overline{N}_l 
\end{bmatrix},
\]
and applying $\begin{bmatrix}
- F^{-1}(x_{l-1}) & 0_{n\times n} \\
0_{n\times n} &  F^{-1}(x_{l}) \\
\end{bmatrix}$ to both sides, we obtain
\[ 
E_{2n\times 2n}
= 
\begin{bmatrix}
- F^{-1}(x_{l-1})\overline{P}_l & - F^{-1}(x_{l-1})P_l \\
F^{-1}(x_{l}) \overline{Q}_l & F^{-1}(x_{l}) Q_l 
\end{bmatrix} 
\begin{bmatrix}
M_l & N_l  \\
\overline{M}_l & \overline{N}_l 
\end{bmatrix}
\]
Note that the left operand on RHS is invertible, so it has full rank. This means that 
\[ H_l := \begin{bmatrix}
-F(x_{l-1})^{-1} P_l \\
F(x_l)^{-1}Q_l
\end{bmatrix}_{n \times (2nk -m)}
\]
has rank $2nk -m.$ Thus, columns $H_l$ also form the solution space of the system \eqref{system}, just like $(H_1)_l.$ But this suggests that $(H_1)_l$ and $H_l$ are the same up to a linear transformation, i.e. there exists a non-singular $(2nk-m)\times (2nk-m)$ matrix $A_l$ such that $(H_1)_l = H_l A,$ i.e.
\[ 
\begin{bmatrix}
-F(x_{l-1})^{-1} (P_1)_l \\
F(x_l)^{-1}(Q_1)_l
\end{bmatrix}  = (H_1)_l = H_l A = \begin{bmatrix}
-F(x_{l-1})^{-1} P_l A\\
F(x_l)^{-1}Q_l A
\end{bmatrix},
\]
and so $(P_1)_l  = P_l A$ and $(Q_1)_l= Q_l A.$ Since $l$ is arbitrary, this holds for all $l = 1,\ldots, k.$ Therefore,
\[ U_1^+ g = \sum^k_{l=1} (P_1)^*_l \vec{g}_l(x_{l-1}) + (Q_1)^*_l \vec{g}_l(x_l) =  \sum^k_{l=1} A^*P_l^* \vec{g}_l(x_{l-1}) + A^*Q_l^*  \vec{g}_l(x_l) = A^*U^+ g.\]
Observe that $U^+ g = \vec{0}$ implies $U_1^+ g = \vec{0}.$ Since $A^*$ is nonsingular, it follows that $U^+ g = \vec{0}$ if and only if $U_1^+ g = \vec{0}.$ Since $U^+ g= \vec{0}$ is adjoint to $Uf= \vec{0}, U^+_1 g= \vec{0}$ is adjoint to $Uf= \vec{0}.$ This completes the proof. 
\end{proof}

\bibliographystyle{plain}
\bibliography{IMVP-Adjoint-Construction-ref}


\end{document}